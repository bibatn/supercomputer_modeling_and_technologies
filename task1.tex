\documentclass[a4paper,12pt,oneside,openleft]{memoir}
\usepackage{graphicx} %package to manage images
\newsubfloat{figure}% Allow subfloats in figure environment
\usepackage[T2A]{fontenc}
\usepackage[utf8]{inputenc}
\usepackage[russian]{babel}
\usepackage{alltt}
\usepackage[T2A]{fontenc}
%\usepackage[koi8-r]{inputenc}

\usepackage[usenames,dvipsnames]{color}

\usepackage{alltt}

\usepackage{listings}

\begin{document}
	
\section{Исходный фрагмент и описание информационной структуры} 
В качестве условия задачи выступает фрагмент программы на языке C, листинг которой приведён
в Приложении 1. Требовалось выполнить исследование информационной структуры этого
фрагмента, то есть выявить имеющиеся в ней зависимости по данным и их характер, после чего
составить описание информационной структуры на языке разметки Algolang. Итоговый листинг
описания структуры фрагмента на языке Algolang получился вот таким:

\lstset{language=XML}
\begin{lstlisting}
<?xml version = "1.0" encoding = "UTF-8"?>
<algo>
	<params>
		<param name = "N" type = "int" value = "5"></param>
		<param name = "M" type = "int" value = "4"></param>
	</params>
		<block id = "0" dims = "1">
		<arg name = "i" val = "3..N+1"></arg>
		<vertex condition = "" type = "1">
			<in src = "i"></in>
		</vertex>
	</block>
	<block id = "1" dims = "1">
		<arg name = "i" val = "2..N+1"></arg>
		<vertex condition = "" type = "1">
			<in src = "i - 1"></in>
			<in bsrc = "0" src = "i"></in>
		</vertex>
	</block>
	<block id = "2" dims = "2">
		<arg name = "i" val = "2..N+1"></arg>
		<arg name = "j" val = "2..M+1"></arg>
		<vertex condition = "" type = "1">
			<in src = "i + 1, j"></in>
		</vertex>
	</block>
	<block id = "3" dims = "3">
		<arg name = "i" val = "2..N+1"></arg>
		<arg name = "j" val = "1..M+1"></arg>
		<arg name = "k" val = "1..N-1"></arg>
		<vertex condition = "(j == 1) and (k == 1)" type = "1">
			<in bsrc = "2" src = "i,M+1"></in>
			<in bsrc = "1" src = "i"></in>
		</vertex>
		<vertex condition = "(j>1)" type = "1">
			<in src = "i-1,j,k-1"></in>
		</vertex>
	</block>
</algo>
\end{lstlisting}

\section{Приложение 1}
Листинг исходного фрагмента на C


\begin{verbatim}

1 for(i = 2; i <= n+1; ++i)
2   C[i] = C[i - 1] + D[i];
3 for(i = 2; i <= n+1; ++i)
4   for(j = 2; j <= m+1; ++j)
5     B[i][j] = B[i + 1][j];
6 for(i = 2; i <= n+1; ++i){
7   A[i][1][1]= B[i][m + 1] + C[i];
8   for(j = 2; j <= m+1; ++j){
9 	  for(k = 1; k <= n; ++k)
10 	    A[i][j][k] = A[i - 1][j][k - 1] + A[i][j][k];
11 	}
12 }
\end{verbatim}



     	
\end{document}